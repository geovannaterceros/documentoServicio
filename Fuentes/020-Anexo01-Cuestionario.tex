\chapter{Cuestionario de identificacion del servicio web}
\label{cuestionario}
Para seleccionar el servicio, se realizan las siguientes preguntas: \\

\textbf{1.-} ¿ El servicio esta asociado, con una solo entidad l'ogica que se usa en diferentes procesos?\\
\textbf{R.-} Si, el servicio, se utiliza un solo archivo que tiene la extensi'on sis, el cual se encuentra en dos procesos en la p'agina del SAGAA.\\
 
\textbf{2.-} ¿ Qu'e operaciones, se deben soportar que, se realizan usualmente sobre dicha entidad?\\
\textbf{R.-} La entidad el archivo con la extensi'on sis, se puede modificar a trav'es de la aplicaci'on del Transcriptor.exe.\\

\textbf{3.-} ¿ Se trata de tareas que realizan diferentes personas en la organizaci'on?\\
\textbf{R.-} Solo lo realiza un usuario el docente.\\

\textbf{4.-} ¿ El servicio es independiente?\\
\textbf{R.-} Si es independiente de las otras funciones. \\

\textbf{5.-}¿ El servicio tiene estado?\\
\textbf{R.-} No tiene estado.\\

\textbf{6.-} ¿ El servicio pueden usarlo, clientes fuera de la organizaci'on?\\
\textbf{R.-} Si lo pueden usar los estudiante, pero no esta habilitado para ellos.\\

\textbf{7.-} ¿ Diferentes usuarios, tienen distintos requerimientos?\\
\textbf{R.-} Los estudiantes, desean ver sus notas, esto se puede ver como requerimientos no funcionales.\\