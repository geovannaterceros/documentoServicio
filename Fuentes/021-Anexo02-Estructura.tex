\chapter{Estructura del Proyecto}
\label{estructuraIonic}
La estructura del proyecto usa la arquitectura de ionic este tiene diferentes componentes  se explican a continuaci'on:

\begin{enumerate}
\item \textbf{Config y routes:} se desarrollan en el archivo app.js donde se realiza la configuraci'on y las rutas que permiten enlazar el controlador con la la interfaz de usuario correspondiente.
\item \textbf{Controller:} es el archivo controller.js,  realiza la comunicaci'on a trav'es de la variable scope entre la interfaz de usuario y los servicios de los archivos services.js o factory.js.
\item \textbf{Directive:} permite crear y usar componentes con aspectos y comportamiento se encuentra en el archivo directive.js .
\item \textbf{Factory:} tiene dos archivos service.js y factory.js. Este modulo ayuda a obtener los datos del servicio.
\item \textbf{View:} son los archivos html que contienen la descripci'on visual y obtiene los datos a mostrar el scope.
\end{enumerate}

El framework ionic, utiliza la estructura de modelo, vista y controlador del proyecto, el c'ual se genera autom'aticamente al momento de crear el proyecto en carpetas y archivos, para organizar el c'odigo en los siguientes archivos, se desarrolla en el anexo 

\begin{verbatim}
 proyecto
    |--bower.json (Lista de dependencias y paquetes de Bower)
    |--.bowerrc
    |--config.xml (Contiene la configuracion de la plataforma)
    |--.editorconfig
    |--.gitignore
    |--gulpfile.js (Lista de tareas de Gulp)
    |--hooks/ (Anade scripts que producen eventos)
    |--ionic.project (Configuracion de Ionic)
    |--package.json (Dependencias y paquetes de NodeJS)
    |--platforms/ (Codigo plataformas para compilar)
    |--plugins/ (Plugins o modulos de aplicacion)
    |--resources/ (Recursos plataforma concreta)
    |--scss/ ( Codigo SCSS compilado en www/css)
    |--www/ (Codigo fuente principal)
        |--css/ (Estilos que se usa en la aplicacion)
        |--img/ (Imagenes de nuestro proyecto)
        |--index.html (Fichero principal, cargamos necesario)
        |--js/ (El codigo, Javascript de la aplicacion)
            |--app.js 
            |--controllers.js
            |--directive.js
            |--factory.js
            |--filter.js
            |--service.js
        |--lib (Librerias, del codigo)
        |--templates/(Vistas de la aplicacion)
\end{verbatim}

\subsection{Configuraci'on del proyecto}

El presente proyecto utiliza el interprete de linea de comando denominado (cli) de ionic el cual tiene los siguientes comandos:

\begin{verbatim}
$ ionic start appSAGAA sidemenu (crea proyecto)
$ ionic platform add android (añadir la plataforma)
$ ionic build android (compilar el proyecto)
$ ionic run android (ejecutar el proyecto)
$ ionic serve (ejecutar, compila y muestra en el navegador )
\end{verbatim}

Se crea autom'aticamente la carpeta de appSAGAA con  la estructura  de la figura \ref{EstructuraIonic} y el men'u sidemenu por defecto.