\chapter{MANUAL DE USUARIO}
\label{ManualU}
\section{Introducci'on}
El presente manual est'a elaborado con la finalidad de brindarle informaci'on acerca del funcionamiento de MultiPile Matrix.

Es de mucha importancia leer este manual antes de utilizar MultiPile Matrix.

\section{Descripci'on}
MultiPile Matrix es una herramienta que proporciona distintas funciones para el an'alisis y la navegaci'on en la evoluci'on de datos (dos criterios relacionados en el tiempo).

\section{MultiPile Matrix}
Para acceder a MultiPile Matrix solo es necesario ejecutar Pharo con la imagen \emph{MultiPileMatrix.image} que se encuentra en el CD, en caso de Windows y Mac OS solo es necesario arrastrar la imagen a la aplicaci'on, mientras que para Linux se debe ejecutar Pharo y como argumento la imagen.

\begin{figure}[h]
    \centering
    \includegraphics[width=0.85\textwidth]{FiguraAG01VentanaMM}
    \caption{Ejecutando MultiPile Matrix. Fuente: Elaboraci'on propia}
    \label{fig:VentanaMM}
\end{figure}

Una vez realizados los anteriores pasos se presenta una ventana como se observa en la figura \ref{fig:VentanaMM}.

\section{Funciones}
MultiPile Matrix presenta funciones para cargar archivo, apilaci'on de tiempos, apilaci'on de objetos en base a su relaci'on, resaltado de objetos y resaltado de relaciones en base a valor. 

\begin{figure}[h]
    \centering
    \includegraphics[width=0.7\textwidth]{FiguraAG02DivisionMM}
    \caption{Partes de la interfaz de MultiPile Matrix. Fuente: Elaboraci'on propia}
    \label{fig:DivisionMM}
\end{figure}

Las distintas funciones de MultiPile Matrix estan organizadas en la interfaz como se muestra en la figura \ref{fig:DivisionMM}, cada una se describe a continuaci'on. 


\subsection{Cargar un archivo}
En primer lugar para analizar datos, se necesita cargar un archivo .csv, como se observa en la figura \ref{fig:DivisionMM} inciso 1), MultiPile Matrix tiene el boton \emph{Load file csv} para realizar esta tarea. El archivo que se desea cargar debe cumplir con las siguientes caracter'isticas:

\begin{itemize}
\item El archivo debe tener como primera fila las cabeceras.
\item El archivo debe tener una columna identificada con la cabecera time y cada valor de esta columna debe ser un numero.
\item El archivo debe tener una columna identificada con la cabecera weigth y cada valor de esta columna debe ser un numero.
\item El archivo debe tener dos columnas m'as aparte de las anteriormente mencionadas, en estas dos columnas se encuentran los objetos que se relacionan.
\end{itemize}

Por lo tanto, el archivo debe tener minimamente 4 columnas, como se observa en la figura \ref{fig:Archivo1}.

\begin{figure}[h]
    \centering
    \includegraphics[width=0.4\textwidth]{FiguraAG03Archivo}
    \caption{Ejemplo de archivo. Fuente: Elaboraci'on propia}
    \label{fig:Archivo1}
\end{figure}

La informaci'on anterior tambi'en se puede observar con el boton \emph{Help with the file} presente en la figura \ref{fig:DivisionMM} inciso 1).
En caso de que el archivo no respete la estructura, no se lo tomara en cuenta.

\subsection{Definir par'ametros}
Se debe definir en que columna estan los valores de: tiempo, objetos inicio, objetos destino y la relaci'on. Para ello se presenta el inciso 2) de la figura \ref{fig:DivisionMM}.

Por defecto esos valores estan de la siguiente forma: tiempo (primera columna), objetos inicio (segunda columna), objetos destino (tercera columna) y relaci'on (cuarta columna). Sin embargo, las otras funciones como apilaci'on y resaltado no se encontraran disponibles hasta escoger debidamente los par'ametros.

Estos par'ametros seran definidos al realizar un clic sobre el boton \emph{Accept}, para asi llenar los datos de acuerdo a los par'ametros para tener la opci'on de utilizar las funciones de apilaci'on y resaltado.

\subsection{Funciones apilaci'on y resaltado}
Para todas las funciones de apilaci'on y resaltado se deben llenar los campos necesarios, una vez llenados esos campos se puede agregar la configuraci'on correspondiente realizando clic en el boton \emph{+}, agregandola en el panel del lado derecho.

Para remover alguna configuraci'on, esta debe ser seleccionada del panel del lado derecho y al realizar un clic en el boton \emph{x}, esta se eliminar'a del panel.

\subsection{Apilaci'on por tiempo}
Esta funci'on es el inciso 3) de la figura \ref{fig:DivisionMM}; permite apilar tiempos de acuerdo al intervalo $[inicio, final]$, por lo cual para apilar por tiempo se necesitan dos campos: el tiempo de inicio y el tiempo final.

\subsection{Apilaci'on por objeto}
Esta funci'on es el inciso 4) de la figura \ref{fig:DivisionMM}; permite apilar los tiempos adyacentes que respetan un criterio en base a un objeto. Para realizar esta funci'on se necesitan dos campos: el criterio y el objeto. 

Para el criterio se puede escoger entre:
\begin{itemize}
\item \emph{contains:} Hace referencia a que todo tiempo que se apile, debe tener una relaci'on mayor a 0 que involucre al objeto escogido.
\item \emph{notContains:} Hace referencia a que todo tiempo que se apile, no debe tener una relaci'on mayor a 0 que involucre al objeto escogido.
\end{itemize}

En caso de tener apilaciones de tiempo se ignora a esas apilaciones y se les da prioridad a las apilaciones por objeto.

En caso de tener varias apilaciones por objeto se realiza las apilaciones de tiempo en base a si un tiempo cumple con cualquiera de los criterios que se encuentran en el panel del lado derecho.

\subsection{Resaltado por objeto}
Esta funci'on es el inciso 5) de la figura \ref{fig:DivisionMM}; permite resaltar con un color todas las relaciones mayores a 0 en las que esta involucrado un objeto, para ello se debe seleccionar dos campos: color y objeto. El color seleccionado actuar'a como el color que representa el valor m'aximo.

\subsection{Resaltado por valor}
Esta funci'on es el inciso 6) de la figura \ref{fig:DivisionMM}; permite resaltar con un color todas las relaciones que satisfacen un criterio, para ello se debe seleccionar tres campos: color, criterio y valor. El color seleccionado actuar'a como el color que representa el valor m'aximo.

Los criterios que se tiene como opciones son:
\begin{itemize}
\item \emph{\textgreater} Hace referencia a las relaciones que son mayores a un valor dado.
\item \emph{\textless} Hace referencia a las relaciones que son menores a un valor dado.
\item \emph{=} Hace referencia a las relaciones que son iguales a un valor dado.
\end{itemize}

Para ambos casos de resaltado si una relaci'on satisface m'as de un criterio; es decir, tiene asignados dos o m'as colores, solo se escoge un color.

\subsection{Visualizar}
Finalmente una vez el archivo esta cargado y la configuraci'on es la deseada, se presenta el boton \emph{Visualize} que es el inciso 7) de la figura \ref{fig:DivisionMM} para mostrar la visualizaci'on requerida.