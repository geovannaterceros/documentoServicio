\usepackage{fancybox}

% Abreviaciones
\newcommand{\umss}[0]{Universidad Mayor de San Sim\'{o}n}
\newcommand{\fcyt}[0]{Facultad de Ciencias y Tecnolog\'{i}a}

\newcommand{\TITULO}[0]{APLICACI\'{O}N M\'{O}VIL EN ANDROID QUE EXTIENDE SERVICIOS DE LA APLICACI\'{O}N SAGAA PARA LA DISPONIBILIDAD DE INFORMACI\'{O}N}
\newcommand{\geovana}[0]{Geovanna Lizette Gil Terceros}
\newcommand{\jorge}[0]{Tutor: Msc. Ing. Orellana Araoz Jorge Walter}

% Definici'on de comandos para formato

\newcommand{\comentarios}[1]{}
\newcommand{\concepto}[1]{\emph{#1}} 
\newcommand{\tips}[1]{\comentarios{#1}}
% hay que poner en el indice los conceptos

\newcommand{\margen}[1]{\marginpar{\small #1}} 
% notas al margen
                                   
\newenvironment{comentario}
              {\em \noindent {Comentario:}\begin{quotation}}
              {\end{quotation}\noindent {Fin Comentario}}
\newcommand{\nota}[1]{\vspace{5mm} {\em \noindent Nota: #1}\\ }
%\newcommand{\nota}[1]{}


\newcommand{\linea}{\noindent \rule{12cm}{0.5mm}}

% Comentarios y notas
\newcounter{comm}
\newcommand{\parrafo}[2]{%
  \ifthenelse{\value{comm}>0}%
   {#1: \emph{#2}}%
   {}%
}
\newcommand{\falta}[1]{\parrafo{Falta}{#1}}
\newcommand{\sugerencia}[1]{\parrafo{Sugerencia}{#1}}
\newcommand{\obs}[1]{\parrafo{Observaci\'{o}n}{#1}}