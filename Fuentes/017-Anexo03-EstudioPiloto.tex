\chapter{Estudio Piloto}
\label{MaterialMM}

Se realizo un estudio piloto con dos personas, el objetivo de este estudio es el de asegurarse que los materiales de aprendizaje, los cuestionarios se entienden y las preguntas tienen respuestas.
Para el material de aprendizaje de MS-Excel no se tienen observaciones, mientras que para el material de aprendizaje de MultiPile Matrix se tuvieron observaciones de redacci'on en su mayoria, tambi'en se realizaron observaciones en la cabecera de los cuestionarios. 

\begin{figure}[h]
    \centering
    \includegraphics[width=0.4\textwidth]{FiguraAC01Rev1}
    \caption{Observaciones a cerca de la primera parte del material de aprendizaje de MultiPile Matrix.}
    \label{fig:Rev1}
        \centering
    \includegraphics[width=0.4\textwidth]{FiguraAC02Rev0}
    \caption{Observaciones a cerca de la parte de pila de matrices del material de aprendizaje de MultiPile Matrix.}
    \label{fig:Rev0}
\end{figure}

\begin{figure}[h]
    \centering
    \includegraphics[width=0.4\textwidth]{FiguraAC03Rev2}
    \caption{Observaciones de la descripci'on de las partes de MultiPile Matrix.}
    \label{fig:Rev2}
\end{figure}

En las figuras \ref{fig:Rev1}, \ref{fig:Rev0} y \ref{fig:Rev2} se aprecian las observaciones acerca del material de aprendizaje de MultiPile Matrix, que para realizar el experimento se arreglaron.

\begin{figure}[h]
    \centering
    \includegraphics[width=0.4\textwidth]{FiguraAC04Rev3}
    \caption{Observaciones de la cabecera de los cuestionarios.}
    \label{fig:Rev3}
\end{figure}

En la figura \ref{fig:Rev3} las observaciones en la cabecera de los cuestionarios que se basa en dar opciones m'as entendibles para escoger. En las figuras \ref{fig:Rev4} y \ref{fig:Rev5} se aprecian las respuestas de cuestionario, el primero posee un total de 3 preguntas correctas y el segundo un total de 4 correctas de las 5, al analizar las respuestas y que pasos se siguio para responder las preguntas, se decide mejorar la redacci'on de algunas preguntas que confunden al usuario a la hora de responder, a pesar de esto ambos participantes respondieron el cuestionario.

\begin{figure}[h]
    \centering
    \includegraphics[width=0.5\textwidth]{FiguraAC05Rev4}
    \caption{Respuestas al cuestionario.}
    \label{fig:Rev4}
\end{figure}

\begin{figure}[h]
    \centering
    \includegraphics[width=0.5\textwidth]{FiguraAC06Rev5}
    \caption{Respuestas al cuestionario.}
    \label{fig:Rev5}
\end{figure}

Para los ambos estudios con los usuarios que se realizaron no se presento error alguno para responder los cuestionarios con MultiPile Matrix. Lo cual significa que la herramienta presenta posee el comportamiento esperado  de acuerdo al material de aprendizaje entregado.