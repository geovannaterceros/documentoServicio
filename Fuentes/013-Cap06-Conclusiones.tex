\chapter{Conclusiones y Reflexiones}
\label{capituloseis}
Algunas recomendaciones:
\begin{enumerate}
\item Deberia, realizar una aplicacion nativa.
\end{enumerate}
Este proyecto ofrece las siguientes contribuciones:
\begin{itemize}
\item Se presenta MultiPile Matrix, un DSL acoplado con matrices apiladas para producir visualizaciones interactivas.
\item Se evalua MultiPile Matrix usando un experimento centrado en el DSL y en la visualizaci'on.
\end{itemize}

Inicialmente se realiza un estudio de los datos con los que se trabaja, encontrando una estructura que los represente y escogiendo como representarla visualmente, para luego seleccionar las t'ecnicas de visualizaci'on que se fueron encontrando mientras se realizaba un estudio de los trabajos relacionados a evoluci'on de grafos. Escogiendo como mejores las siguientes t'ecnicas: yuxtaposici'on, animaci'on, resumen; con la representaci'on visual de una matriz y la met'afora que consta de apilar matrices.

Tambi'en se distingue los elementos que componen a MultiPile Matrix de acuerdo a la representaci'on y las t'ecnicas seleccionadas. Realizando el dise'no para cada elemento, y luego implementandolo con pruebas unitarias, realizando as'i mejoras en el dise'no y en la implementaci'on a medida que se avanzaba en 'este proyecto. Y modificando su funcionalidad por las caracter'isticas agregadas a 'ultimo momento, como ser: el resaltado de peso y el filtro de columnas y filas.

Finalmente para la evaluaci'on de MultiPile Matrix, se realiz'o un experimento con 8 personas descrito en el cap'itulo \ref{capitulocinco}, y cuyos resultados demuestran que los participantes tienen un mejor desempe'no con MultiPile Matrix que utilizando Excel para dos conjuntos de datos diferentes; a pesar, de ser la primera vez programando en Pharo y que Excel es una herramienta que conoc'ian y utilizaron durante a'nos. Concluyendo as'i que MultiPile Matrix representa una mejor alternativa que Excel para la navegaci'on y el an'alisis de los dos conjuntos de datos presentados en el experimento.

Obteniendo as'i el desarrollo de una met'afora visual llamada MultiPile Matrix, que permite la navegaci'on visual de la evoluci'on de datos emergentes del desarrollo de software basados en dos criterios relacionados, permitiendo el an'alisis de datos con al facilidad de resaltar caracter'isticas y comparar tiempos.

Como trabajo futuro se tomo en cuenta la retrospectiva descrita en el cap'itulo \ref{capitulocinco}, en la cual se recomienda trabajar con repositorios de GitHub.